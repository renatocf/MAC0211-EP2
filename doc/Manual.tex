\documentclass[a4paper,12pt]{article}

% --------------------------------------------------------------------------- %
\usepackage[utf8]{inputenc}             % Encodamento utf-8
\usepackage[brazil]{babel}              % Parcote para texto em português

\usepackage{setspace}                   % Espaçamento flexível
\usepackage{indentfirst}                % Indentação do primeiro parágrafo
\usepackage[fixlanguage]{babelbib}      % Opções extras para linguagem

\usepackage[usenames,svgnames,dvipsnames]{xcolor}
\usepackage[font=small,format=plain,labelfont=bf,up,textfont=it,up]{caption}
\usepackage[a4paper,top=2.54cm,bottom=2.0cm,left=2.0cm,right=2.54cm]{geometry}

\usepackage[pdftex,plainpages=false,pdfpagelabels,pagebackref,colorlinks=true,citecolor=DarkGreen,linkcolor=NavyBlue,urlcolor=DarkCornflowerBlue,filecolor=green,bookmarksopen=true]{hyperref} 				% links coloridos
\usepackage[all]{hypcap}                % soluciona o problema com o hyperref 
					% e capitulos

\newpage %%%%%%%%%%%%%%%%%%%%%%%%%%%%%%%%%%%%%%%%%%%%%%%%%%%%%%%%%%%%%%%%%%%%%%
  \pagenumbering{arabic}     % começamos a numerar 
  \begin{document}

  \begin{center} 
	{\LARGE \textcolor{NavyBlue}{ \textbf{Manual de usuário - Jogo das Canoas}}}
  \end{center}

  \bigskip
  \bigskip

  O Jogo das Canoas consiste, nessa fase inicial, em gerar um rio com 
  características atribuídas pelo próprio usuário, sendo essas características 
  frequência e tamanho de ilhas e velocidade, largura, altura e tamanho máximo 
  das margens do rio. Sendo como se este pudesse alterar a dificuldade do jogo.


\newpage %%%%%%%%%%%%%%%%%%%%%%%%%%%%%%%%%%%%%%%%%%%%%%%%%%%%%%%%%%%%%%%%%%%%%%
\section{\textcolor{NavyBlue}{Compilação}}

Para compilar o jogo, entre na pasta onde o seu jogo e utilize o seguinte comando no terminal:

\$ \textcolor{CornflowerBlue}{\textit{make}}

\bigskip
\section{\textcolor{NavyBlue}{Execução}}

  Para inicializar o jogo, após compilá-lo, digite no próprio terminal:
  
  \$ \textcolor{CornflowerBlue}{\textit{./bin/ep2}}
  
  Digitando apenas o solicitado acima, o jogo terá caracteríscaticas com valores 
  default, atribuídos pelo programador. Caso você queira configurar as características
  deste, acrescente os comandos abaixo ao comando acima. Será mostrado um exemplo para
  auxiliar o compreendimento.


  \bigskip
  \subsection{\textcolor{NavyBlue}{Características adicionais}}

  O usuário poderá modificar as seguintes características via linha de comando:

  \bigskip
  \begin{itemize}
  
  \item Fluxo do rio . . . . . . . . . . . . . . . . . . .  \textcolor{CornflowerBlue}{-F}
  \item Altura do rio. . . . . . . . . . . . . . . . . . .  \textcolor{CornflowerBlue}{-H}
  \item Largura do rio . . . . . . . . . . . . . . . . . .  \textcolor{CornflowerBlue}{-L}
  \item Frequência das ilhas . . . . . . . . . . . . . . .  \textcolor{CornflowerBlue}{-i} 
  \item Distância de segurança entre ilhas. . . . . . .  \textcolor{CornflowerBlue}{-f}
  \item Semente geradora de aleatoriedade . . . . . .  \textcolor{CornflowerBlue}{-s}
  
  \end{itemize}
  \bigskip
  
  Observação: Existe uma opção adicional \textcolor{CornflowerBlue}{-h}, caso você precise de AJUDA.
  
  \newpage %%%%%%%%%%%%%%%%%%%%%%%%%%%%%%%%%%%%%%%%%%%%%%%%%%%%%%%%%%%%%%%%%%%%
  \subsection{\textcolor{NavyBlue}{Exemplos de utilização:}}
  \bigskip
  
    \subsubsection{\textcolor{NavyBlue}{Exemplo 1}}
    \bigskip
    \begin{itemize}
  
    \item Altura do rio = 30  . . \textcolor{CornflowerBlue}{-H30}
    \item Largura do rio = 60  . . \textcolor{CornflowerBlue}{-L60}  
    \item Distância mínima entre as margens = 35  . . \textcolor{CornflowerBlue}{-Z20}
    
    \end{itemize}  
    \bigskip
    
    \$ \textcolor{CornflowerBlue}{\textit{make}}
    
    \$ \textcolor{CornflowerBlue}{\textit{./bin/ep2 -H30 -L60 -Z20}}
  
    \bigskip
    \begin{verbatim}
    #########..........................................#########
    #########.........................................##########
    ########...........................................#########
    #########.........................................##########
    ########.........................................###########
    ########.........................................###########
    #########.........................................##########
    ##########.......................................###########
    ###########......................................###########
    ###########.........................############.###########
    ############......................................##########
    ###########.......................................##########
    ##########.........................................#########
    #########...........................................########
    #########..........................................#########
    ##########........................................##########
    #########..........................................#########
    ########..........................................##########
    #######............................................#########
    ########..........................................##########
    ########..........................................##########
    #######.......................##########.........###########
    #######.........................................############
    ######..........................................############
    #######.........................................############
    #######..........................................###########
    ######...........................................###########
    #####.........########............................##########
    #####..............................................#########
    #####..............................................#########
    \end{verbatim}
  
  \newpage %%%%%%%%%%%%%%%%%%%%%%%%%%%%%%%%%%%%%%%%%%%%%%%%%%%%%%%%%%%%%%%%%%%%
    \subsubsection{\textcolor{NavyBlue}{Exemplo 2}}
    
    \begin{itemize}
    \bigskip
    
    \item Fluxo do rio = 20  . . \textcolor{CornflowerBlue}{-F20}  
    \item Altura do rio = 25  . . \textcolor{CornflowerBlue}{-H25}
    \item Largura do rio = 65  . . \textcolor{CornflowerBlue}{-L65}
    \item Frequência das ilhas = 80\%  . . \textcolor{CornflowerBlue}{-i0.8}
    \item Distância mínima entre ilhas = 2  . . \textcolor{CornflowerBlue}{-f2}
    \item Semente geradora de aleatoriedade = 10  . . \textcolor{CornflowerBlue}{-s10}
    \item Distância mínima entre as margens = 35  . . \textcolor{CornflowerBlue}{-Z35}
    
    \end{itemize}  
    \bigskip
    
    \$ \textcolor{CornflowerBlue}{\textit{make}}
    
    \$ \textcolor{CornflowerBlue}{\textit{./bin/ep2 -F20 -H25 -L65 -i0.8 -f2 -s10 -Z35}}
    
    \bigskip
    \begin{verbatim}
    #.......................###......................#
    #................................................#
    ##..............................................##
    ##........##########............................##
    #...............................................##
    ##...............................................#
    ##...................#################..........##
    ##..............................................##
    ###..............................................#
    ####............................###############.##
    #####...........................................##
    ####............................................##
    #####..........#########.........................#
    #####............................................#
    #####...........................................##
    #####............................................#
    ######..........................................##
    ######...........................................#
    #######.......##################.................#
    #######..........................................#
    ######...........................................#
    #####........##########..........................#
    ####.............................................#
    #####...........................................##
    #####...........................................##
    \end{verbatim}

\end{document}

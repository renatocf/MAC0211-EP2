%\documentclass[twoside]{article}
\documentclass[a4paper,12pt]{article}

% --------------------------------------------------------------------------- %
\usepackage[utf8]{inputenc}             % Encodamento utf-8
\usepackage[brazil]{babel}              % Parcote para texto em português

\usepackage{setspace}                   % Espaçamento flexível
\usepackage{indentfirst}                % Indentação do primeiro parágrafo
\usepackage[fixlanguage]{babelbib}      % Opções extras para linguagem

\usepackage[usenames,svgnames,dvipsnames]{xcolor}
\usepackage[font=small,format=plain,labelfont=bf,up,textfont=it,up]{caption}
\usepackage[a4paper,top=2.54cm,bottom=2.0cm,left=2.0cm,right=2.54cm]{geometry}

\usepackage[pdftex,plainpages=false,pdfpagelabels,pagebackref,colorlinks=true,citecolor=DarkGreen,linkcolor=NavyBlue,urlcolor=DarkRed,filecolor=green,bookmarksopen=true]{hyperref} 				% links coloridos
\usepackage[all]{hypcap}                % soluciona o problema com o hyperref 
					% e capitulos

\newpage %%%%%%%%%%%%%%%%%%%%%%%%%%%%%%%%%%%%%%%%%%%%%%%%%%%%%%%%%%%%%%%%%%%%%%
\pagenumbering{arabic}     % começamos a numerar 
\begin{document}
 
\begin{center} 
	{\LARGE Manual de usuário - Jogo das Canoas}
\end{center}



\bigskip
\bigskip


O Jogo das Canoas consiste, nessa fase inicial, em gerar um rio com características atribuídas pelo próprio usuário, sendo essas características frequencia e tamanho de ilhas e velocidade, largura, altura e tamanho máximo das margens do rio.
Sendo como se este pudesse alterar a dificuldade do jogo.

 
 \newpage
\begin{center}
{\Large Compilação do jogo:}
\end{center}

Para compilar o jogo, entre na pasta onde o seu jogo e utilize o seguinte comando no terminal:

\$: make

\begin{center}
{\Large Execução do jogo:}
\end{center}

\begin{center}
 {\Large Características adicionais:}
\end{center}

{\large O usuário poderá modificar as seguintes características via linha de comando:}

\bigskip
\begin{itemize}
\item Fluxo do rio . . . . . . . . . . . . . .  -f
\item Altura do rio. . . . . . . . . . . . . .  -H 
\item Largura do rio . . . . . . . . . . . . .  -L
\item Velocidade do rio. . . . . . . . . . . .  -v	
\item Tamanho das ilhas. . . . . . . . . . . .  -
\item Frequência das ilhas . . . . . . . . . .  -I
\item Posição mínima da margem direita . . . .  -r 
\item Posição máxima da margem esquerda. . . .  -l
\item Semente geradora de aleatoriedade. . . .  -s
\end{itemize}

\bigskip

Observação: Existe uma opção adicional -h, caso você precise de AJUDA.
 

\end{document}
